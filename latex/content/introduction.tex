
\section{Introduction}

Depression presents a significant global health challenge, affecting millions of individuals worldwide with symptoms such as persistent low mood, loss of interest, and lack of energy \cite{peveler2002depression, wu2020using}. Despite its importance, treatment rates remain low due to barriers like cost, time, and patient reluctance to disclose their mental state during clinical assessments \cite{kessler2012costs}. To address these challenges, automated depression detection systems have emerged to enable private self-assessment and encourage consultation with psychologists.

Previous studies have explored various approaches, including feature extraction methods and machine learning techniques \cite{sun2017random, yang2016decision, gong2017topic}, with recent advances focusing on deep learning approaches that integrate multi-modal features, such as deep convolutional neural network (CNN) \cite{yang2017multimodal}, LSTM \cite{al2018detecting}, and CNN-LSTM \cite{ma2016depaudionet} networks. By integrating advanced feature extraction and deep learning techniques, researchers aim to develop robust systems for depression assessment that overcome the limitations of traditional diagnostic methods.

Throughout this study, we aim to use the Emotional Audio-Textual Depression Corpus (EATD-Corpus) \cite{shen2022automatic} dataset to explore the possibility of detecting the depression status of the individuals using solely their acoustic features extracted from the audio recording from their interview and classify them as depressed and non-depressed groups. To achieve this, we will attempt to recreate the results reported in the original EATD-Corpus paper and compare them with our experiments. Our investigations include the utilization of a convolutional neural network (CNN) coupled with Log-Mel spectrogram features, a CNN-LSTM network employing extracted MFCC features, and a set of traditional classifiers trained on custom statistical features.

Section~\ref{sec:methods} contains all the details regarding the dataset, preprocessing, feature extraction methods, data augmentation and resampling, classification methods investigated, and the evaluation metrics. Section~\ref{sec:results} will discuss the effectiveness of our experiments compared to those reported by EATD-Corpus. In Section~\ref{sec:discussion}, we will address the limitations of our study and propose potential improvements for future research. Finally, Section~\ref{sec:conclusion} summarizes our study to highlight the key findings.